\documentclass[11pt, twoside]{book}
\usepackage[letterpaper, includehead]{geometry}
\usepackage[spanish]{babel}
\usepackage{graphicx}
\graphicspath{{./images/}}
\usepackage[table]{xcolor}
\usepackage{fancyhdr}
\usepackage{lmodern}
\usepackage{enumerate, multicol}
\usepackage{enumitem}

\usepackage{emptypage}
\usepackage{hyperref}
\usepackage{tikz}
\usepackage{pgfplots}
\usepackage[most]{tcolorbox}
\usepackage{imakeidx}
\usepackage{eso-pic}
\usepackage{titlesec}
\usepackage{colortbl}
\usepackage{csquotes}
\usepackage{booktabs}
\usepackage{subcaption}

%% configs and definitions


%colors
\definecolor{ustgreen}{HTML}{00685a}
\definecolor{ustmidgreen}{HTML}{009b87}
\definecolor{ustlightgreen}{HTML}{00b49d}
\definecolor{ustorange}{HTML}{ff6d3c}
\definecolor{ustdarkgreen}{HTML}{004d43}
\definecolor{graytitle}{HTML}{69827f}
\definecolor{grayheader}{HTML}{5c706d}
\definecolor{darkgraytitle}{HTML}{151a19}
\definecolor{lightbackground}{HTML}{bae7e2}
\definecolor{ustblue}{HTML}{006db6}
\definecolor{ustlightblue}{HTML}{69c0d3}
\definecolor{ustdarkblue}{HTML}{003d6b}
\definecolor{ustskyblue}{HTML}{4fa8c2}
\definecolor{usttealblue}{HTML}{006f8e}
\definecolor{ustred}{HTML}{d94c4c}
\definecolor{ustdarkred}{HTML}{9e1b32}
\definecolor{ustlightred}{HTML}{ff8074}
\definecolor{ustbrickred}{HTML}{9b3d3d}
\definecolor{ustscarlet}{HTML}{f44e42}
\definecolor{tablemid}{RGB}{232,251,247}
\definecolor{tabletop}{RGB}{192, 204, 202}

%format of the title for chapters
\titleformat{\chapter}[display]
{\bfseries\large}
{\LARGE\textcolor{graytitle}{Unidad \thechapter}\\* \titlerule}
{2pt}
{\huge \centering\color{darkgraytitle}\titlerule[2pt]\vspace{0.85cm}}
[\vspace{0.85cm}{\titlerule[2pt]}]
\titlespacing*{\chapter}{0pt}{-40pt}{40pt}

% fancyhdr
\newcommand{\numberedchapter}{
    \cleardoublepage
    \fancyhead[RE]{{\bfseries \leftmark}}
    \fancyhead[LO]{{\bfseries \rightmark}}
}
\newcommand{\unnumberedchapter}[1]{
    \cleardoublepage
    \addcontentsline{toc}{chapter}{#1}
    \fancyhead[RE]{{\bfseries #1}}
    \fancyhead[LO]{{}}
}
\pagestyle{fancy}
\fancyhf{}
\fancyhead[LE,RO]{logo}
\fancyhead[RE]{\large\color{grayheader} Departamento de Ingeniería}
\fancyfoot[LO]{\small\color{grayheader} Dr. José Alejandro Aburto}
\fancyfoot[LE,RO]{\color{grayheader} \thepage}

% hyperref setup
\hypersetup{
    colorlinks=true,
    linkcolor=darkgraytitle,
    filecolor=black!30!magenta,
    urlcolor=black!40cyan,
    bookmarks=true,
    pdfauthor={Dr. José Alejandro Aburto Araneda}
}

%index
\makeindex[columns=2]

% thm & tcolorboxes numbering
\newtheorem{theorem}{Teorema}[chapter]
\newtcolorbox[auto counter, number within=chapter]{definition}[2][]{colback=ustblue!2, colframe=ustblue, fonttitle=\bfseries, title=Definición~\thetcbcounter \hfill \textcolor{ustblue!10}{ #2 #1}}
\newtcolorbox[auto counter, number within=chapter]{example}[1][]{colback=ustgreen!5, colframe=ustgreen, fonttitle=\bfseries, #1, title=Ejemplo~\thetcbcounter, breakable}
\newtcolorbox[auto counter, number within=chapter]{examples}[1][]{colback=ustgreen!5, colframe=ustgreen, fonttitle=\bfseries, #1, title=Ejemplos~\thetcbcounter, breakable}
\newtcolorbox{note}{enhanced, colback=ustred!7, frame style={left color=ustdarkred, right color=ustred}, fonttitle=\bfseries, sharp corners, title=Observación, watermark tikz={\draw[line width=2mm] circle (1cm)
node{\fontfamily{ptm}\fontseries{b}\fontsize{20mm}{20mm}\selectfont !};}}
\newtcolorbox{simpleframe}{colframe=ustbrickred, float, enhanced}
\begin{document}
% titlepage
\begin{titlepage}
    \newgeometry{left=7.5cm}
    \noindent
    \begin{flushright}
        \color{white}{\fontsize{32}{54} \selectfont \textsf{Matemáticas para Ciencias e Ingeniería I}}
    \end{flushright}

    \makebox[0pt][l]{\rule{1.3\textwidth}{1pt}}
    \par
    \pagecolor{ustmidgreen}

    \noindent {\Large \textcolor{ustdarkgreen}{\textsf{Universidad Santo Tomás}}\ \textcolor{ustlightgreen!10}{\textsf{Facultad de Ingeniería}}}
    \vfill
    \noindent
    \textcolor{ustlightgreen!15}
    {\Large \textsf{Departamento de Ingeniería}}
    \vskip\baselineskip
    \noindent
    \textcolor{white}{\textsf{2025}}
\end{titlepage}

\restoregeometry
\nopagecolor

\tableofcontents

\chapter*{Presentación}
\addcontentsline{toc}{chapter}{Presentación}
Este apunte busca ser una guía esencial y concisa para el curso de primer año \emph{Matemáticas para Ciencias e Ingeniería I} que comparten diversas carreras en la Universidad Santo Tomás.


Incluye el contenido en pocas palabras, incluyendo ejemplos y ejercicios propuestos. En versiones posteriores se pretende agregar:
\begin{enumerate}[label=\roman*)]
    \item Ejercicios resueltos, con mayor complejidad que los ejemplos básicos del contenido
    \item Agregar un solucionario para los ejercicios
    \item Añadir una sección o subsección en cada capitulo con ideas o problemas interesantes para profundizar
    \item Añadir una bibliografía
\end{enumerate}

\chapter{Lógica y Polinomios}
\section{Lógica proposicional}
Comencemos con una definición fundamental. Queremos formalizar el concepto de \emph{proposición}, las cuales trabajaremos usualmente como incógnitas y operaremos con operaciones binarias (que requieren dos) semejantes a la suma o al producto usuales.
\begin{definition}{Proposición}
    Una proposición es una expresión que puede poseer solo uno de dos valores: \textbf{verdadero} ó \textbf{falso}.
\end{definition}
Podemos comenzar pensando en qué no es una proposición, y hay muchísimos ejemplo: todas las preguntas no son proposiciones. Primero tenemos proposiciones en el lenguaje cotidiano, y también expresiones matemáticas.
\begin{example}
    \begin{enumerate}
        \item Todos los alumnos de la UST deben sacar \(5.5\) o más para eximirse.
        \item El número \(5\) es par.
        \item La ecuación \(x^2+x+1\) no tiene solución en los números reales.
        \item Todos los gatos son grises.
    \end{enumerate}
\end{example}


Tendremos tres operaciones:
\begin{itemize}[label=\(\bullet\)]
    \item La conjunción, \enquote{y}, que denotaremos por  \(\wedge\)
    \item La disyunción, \enquote{ó}, que denotaremos por \(\vee\)
    \item La negación, \enquote{no}, que denotaremos por \(\neg\)
\end{itemize}
\begin{note}
Hay 3 notaciones para la negación, por ejemplo:
    \[\neg p=\sim p =\overline{p}\]
\end{note}
\section{Tablas de verdad}
Como las proposiciones solo pueden tener dos valores: \textbf{verdadero} ó \textbf{falso}, podemos describir los resultados de las operaciones al usarlas en una tabla. Este tipo de tabla les llamaremos \textbf{tablas de verdad}. Las siguientes tablas son por definición:
\begin{table}[h]
    \centering
    \begin{subtable}{0.3\linewidth}\centering
            \begin{tabular}{ccc}
                \toprule
                \(p\)   &   \(q\)   &   \(p\wedge q\) \\
                \midrule
                \(V\)   &   \(V\)   &   \(V\) \\
                \rowcolor{tablemid}
                \(V\)   &   \(F\)   &   \(F\) \\
                \(F\)   &   \(V\)   &   \(F\) \\
                \rowcolor{tablemid}
                \(F\)   &   \(F\)   &   \(F\) \\
            \end{tabular}
            \caption{Conjunción}
    \end{subtable}
    \hfill
         \begin{subtable}{0.3\linewidth}\centering
            \begin{tabular}{ccc}
                \toprule
                \(p\)   &   \(q\)   &   \(p\vee q\) \\
                \midrule
                \(V\)   &   \(V\)   &   \(V\) \\
                \rowcolor{tablemid}
                \(V\)   &   \(F\)   &   \(V\) \\
                \(F\)   &   \(V\)   &   \(V\) \\
                \rowcolor{tablemid}
                \(F\)   &   \(F\)   &   \(F\) \\
            \end{tabular}
            \caption{Disyunción}
    \end{subtable}
    \hfill
    \begin{subtable}{0.3\linewidth}\centering
            \begin{tabular}{cc}
                \toprule
                \(p\)   &   \(\overline{p}\) \\
                \midrule 
                \(V\)   &   \(F\)   \\
                \rowcolor{tablemid}
                \(F\)   &   \(V\)
            \end{tabular}
            \caption{Negación}
\end{subtable}
\end{table}


Podemos utilizar las tablas de verdad para determinar los valores posibles de una proposición compuesta, es decir, una proposición construida con estos operadores usando otras proposiciones. Por ejemplo:
\begin{example}
    Consideremos la proposición compuesta: \((p\vee q)\wedge (\overline{p}\vee\overline{q})\). Sería difícil desarrollarlo en solo un paso, por lo que agregaremos columnas a la tabla de manera que nos ayude a calcular el resultado, poniendo partes más simples que forman la proposición compuesta que nos interesa:
    \begin{center}
        \begin{tabular}{cccc}
            \toprule
            \(p\)   &   \(q\)   &    \(p\vee q\)    &  \(\overline{p}\vee\overline{q}\) \\
            \midrule
            \(V\)   &   \(V\)   &   \(V\)           & \(F\) \\
            \rowcolor{tablemid}
            \(V\)   &   \(F\)   &   \(V\)           & \(V\)\\
            \(F\)   &   \(V\)   &   \(V\)           &\(V\)\\
            \rowcolor{tablemid}
            \(F\)   &   \(F\)   &   \(F\)           & \(V\)
        \end{tabular}
    \end{center}
Ahora que ya hemos calculado estas proposiciones más simples, podemos calcular los valores posibles de la proposición compuesta inicial:
    \begin{center}
        \begin{tabular}{ccccc}
            \toprule
            \(p\)   &   \(q\)   &    \(p\vee q\)    &  \(\overline{p}\vee\overline{q}\) & \((p\vee q)\wedge (\overline{p}\vee\overline{q})\)\\
            \midrule
            \(V\)   &   \(V\)   &   \(V\)           & \(F\) & \\
            \rowcolor{tablemid}
            \(V\)   &   \(F\)   &   \(V\)           & \(V\) &\\
            \(F\)   &   \(V\)   &   \(V\)           &\(V\)  &\\
            \rowcolor{tablemid}
            \(F\)   &   \(F\)   &   \(F\)           & \(V\) & 
        \end{tabular}
    \end{center}
\end{example}

\begin{definition}{Implica}
    Definimos el conectivo (u operador)	
\end{definition}
\end{document}
